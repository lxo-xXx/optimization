% Discussion and Practitioner Guidelines
\section{Discussion and Practitioner Guidelines}
\label{sec:discussion}
Table~\ref{tab:guidelines} summarizes trade-offs to guide method selection.

\begin{table}[h]
  \centering
  \caption{Method selection guidelines (illustrative).}
  \label{tab:guidelines}
  \begin{tabular}{@{}llll@{}}
    \toprule
    Problem & Recommended methods & Strengths & Caveats \\
    \midrule
    Smooth, well-conditioned & GD, Nesterov & Simple, scalable & Sensitive to step size \\
    Ill-conditioned & L-BFGS, preconditioned GD & Fast convergence & Memory, Hessian-vector products \\
    Composite (e.g., LASSO) & ISTA/FISTA, proximal L-BFGS & Structure-exploiting & Prox operator must be efficient \\
    Constrained (convex) & Interior-point, projected gradient & Robust, accurate & Per-iter cost can be high \\
    Large-scale stochastic & SGD, Adam, SVRG/SAGA & Scalable, parallelizable & Tuning, generalization \\
    Distributed/decomposable & ADMM, proximal gradient & Decomposition, ADMM flexibility & Communication overhead \\
    \bottomrule
  \end{tabular}
\end{table}

In practice, initialization, scaling, and diagnostics matter as much as the nominal algorithmic choice. Monitor stationarity, feasibility, and objective trends; adjust step sizes or trust radii; and prefer restart strategies for accelerated methods when oscillations occur.